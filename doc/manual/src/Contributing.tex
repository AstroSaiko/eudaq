% !TEX root = EUDAQUserManual.tex
\section{Developing and Contributing to EUDAQ}

If you would like to contribute your code back into the main repository, please follow the ``fork \& pull request'' strategy:

\begin{itemize}
\item Create a user account on github, log in
\item ``Fork'' the (main) project on github (using the button on the page of the main repo)
\item \emph{Either}: clone from the newly forked project and add
  'upstream' repository to local clone (change user names in URLs
  accordingly):
  \begin{listing}[mybash]
git clone https://github.com/hperrey/eudaq eudaq
cd eudaq
git remote add upstream https://github.com/eudaq/eudaq.git
\end{listing}
\item \emph{or} if edits were made to a previous checkout of upstream: rename origin to upstream, add fork as new origin:

  \begin{listing}[mybash]
cd eudaq
git remote rename origin upstream
git remote add origin https://github.com/hperrey/eudaq
git remote -v show
\end{listing}
\item Optional: edit away on your local clone! But keep in sync with
  the development in the upstream repository by running
  \begin{listing}
git fetch upstream        # download named heads or tags
git pull upstream master  # merge changes into your branch
\end{listing}
on a regular basis. Replace \texttt{master} by the appropriate branch if you work on a separate one.
Don't forget that you can refer to issues in the main repository anytime by using the string \texttt{eudaq/eudaq\#XX} in your commit messages, where \texttt{XX} stands for the issue number, e.g.
  \begin{listing}[mybash]
[...]. this addresses issue eudaq/eudaq#1
\end{listing}
\item Push the edits to origin (our fork)
  \begin{listing}[mybash]
git push origin
\end{listing}
(defaults to \texttt{git push origin master} where origin is the repo and master the branch)
\item Verify that your changes made it to your github fork and then click there on the ``compare \& pull request'' button
\item Summarize your changes and click on ``send''
\item Thank you!
\end{itemize}
