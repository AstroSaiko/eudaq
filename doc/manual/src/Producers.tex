% !TEX root = EUDAQUserManual.tex
\section{Writing a Producer}\label{sec:Producers}
In order to integrate a \gls{DUT} fully into the DAQ, it needs its own Producer.
A Producer is both a CommandReceiver and a DataSender,
meaning it receives commands from Run Control,
and it also sends events to the Data Collector.
A base class is provided that users may inherit from, to make this as easy as possible.
For example code, see \autoref{sec:ExampleProducer}.

\subsection{Configuration}
The \texttt{Configuration} class is a way of storing configuration information
in a way that is easily accessible, and can be saved to or loaded from a human-readable file
(see \autoref{sec:ConfigFiles}), and can be sent over the network.
It is defined in the following header:

\begin{listing}
#include "eudaq/Configuration.hh"
\end{listing}

The configuration consists of a number of sections,
each of which contains a list of name-value pairs.
The values are stored as strings, but they can be converted to/from arbitrary types.
Methods are provided to load from or save to file, to set the current section,
and to set or get configuration values.
An example use is shown below:

\begin{listing}
std::ifstream infile("../conf/ExampleConfig.conf");
eudaq::Configuration config(infile, "Producer.Example");
int param = config.Get("Parameter", 0);
std::cout << "Loaded config, param = " << param << std::endl;
config.Set("Parameter", param+1);
config.Set("OtherParam", "something");
std::ofstream outfile("Test.conf");
config.Save(outfile);
\end{listing}

This creates a configuration loaded from the file \texttt{../conf/ExampleConfig.conf},
selecting the \texttt{Producer.Example} section.
It then gets an integer parameter from the configuration and displays it.
Then it modifies the value of the parameter and sets another parameter,
before writing the configuration to the file \texttt{Test.conf}.

A configuration object will be received by the Producer during the configuration,
as described in \autoref{sec:OnConfigure}.

\subsection{Receiving Commands}
Whenever a command is received from the Run Control,
a corresponding member function of the Producer will be called by the code in the base classes.
In order to react to a command, the necessary code is simply put inside the corresponding method.
The \texttt{Producer} base class is declared by including the following header file:
\begin{listing}
#include "eudaq/Producer.hh"
\end{listing}

\subsubsection{OnInitialise}
This method is called whenever an initialise command is received from the Run Control.
The method signature is:
\begin{listing}
virtual void OnInitialise(const eudaq::Configuration & config);
\end{listing}

Configurations can be set up in the RunControlGUI. They may be used for initialising software.

If you don't need the initialisation step you can simply skip it. In that case a corresponding function of the \texttt{CommandReceiver} base class will be called.

\subsubsection{OnConfigure}\label{sec:OnConfigure}
This method is called whenever a configure command is received from the Run Control.
The method signature is:
\begin{listing}
virtual void OnConfigure(const eudaq::Configuration & config);
\end{listing}

As a parameter, it receives the configuration chosen in the Run Control.
Information may be extracted from the configuration in order to set up the hardware.

\subsubsection{OnStartRun}
This is called on the start of each run.
The method signature is:
\begin{listing}
virtual void OnStartRun(unsigned param);
\end{listing}

As a parameter, it receives the run number of the started run.
The Producer must send a \gls{BORE},
and then prepare for reading out events from the hardware.

\subsubsection{OnStopRun}
This is called at the end of the run.
The method signature is simply:
\begin{listing}
virtual void OnStopRun();
\end{listing}

Care should be taken that there are no more events pending to be read out.
Once all data events have been sent, an \gls{EORE} should also be sent,
to signal to the DAQ that the Producer has ended the run successfully.

\subsection{Sending Data and the RawDataEvent class}
Events may be sent to the DAQ using the \texttt{Producer}'s \texttt{SendEvent()} method
that has the following signature:
\begin{listing}
void SendEvent(const Event &);
\end{listing}

It takes as a parameter an object derived from the \texttt{eudaq::Event} base class
that will be serialised and sent to the Data Collector.
In practice it will usually be of concrete type \texttt{RawDataEvent}.

The \texttt{RawDataEvent} is a generic container for blocks of raw bytes,
used to encapsulate the data read out from the sensor electronics
and send it to the DAQ.
Each \texttt{RawDataEvent} may contain any number of raw data blocks.
By convention each block usually corresponds to one sensor,
but this is not required;
it is up to each Producer how the raw data are encoded,
since it is up to the corresponding DataConverterPlugin how they are decoded.
The RawDataEvent class is defined in the following header file:
\begin{listing}
#include "eudaq/RawDataEvent.hh"
\end{listing}

The class is described in more detail below.

\subsubsection{Constructing}
A \texttt{RawDataEvent} is constructed as follows:
\begin{listing}
RawDataEvent event("EXAMPLE", run, event);
\end{listing}

Where \texttt{"EXAMPLE"} is a string unique to the particular producer
that will be used to select the correct converter during decoding.
The \texttt{run} and \texttt{event} parameters are the run number
and event number respectively.

As well as normal data events, the producer must also send a \gls{BORE}
and \gls{EORE} at the beginning and end of a run respectively.
These are just normal \texttt{RawDataEvent} objects, but with a particular flag set.
The \texttt{RawDataEvent} has factory methods to simplify these cases:
\begin{listing}
RawDataEvent::BORE("EXAMPLE", run);
RawDataEvent::EORE("EXAMPLE", run, event);
\end{listing}

These methods return a \texttt{RawDataEvent} that may be either be sent directly to the DAQ,
or be modified first, e.g. by setting tags as described below in \autoref{sec:Tags}.

\subsubsection{Adding Data}
Once a RawDataEvent has been constructed,
data blocks may be added either using a vector:
\begin{listing}
std::vector<unsigned char> buffer = ...;
event.AddBlock(id, buffer);
\end{listing}

or using a pointer to a block of memory, and a length in bytes:
\begin{listing}
unsigned char * buffer = ...;
event.AddBlock(id, buffer, len);
\end{listing}

Where \texttt{id} is an integer used to differentiate the different blocks.
Usually it can just be 0 for the first block and increment by 1 for the following blocks.
And \texttt{buffer} contains the actual data for the block.
If the buffer is a \texttt{vector}, the whole length is used,
if it is a pointer, then the length must be specified.

The type of the vector or pointer need not be \texttt{unsigned char},
since these methods are in fact template methods that can take a vector of any basic type,
but if larger types are used, care must be taken about endianness,
since the buffer will be converted to \texttt{unsigned char}
according to the endianness of the machine it is running on.
Therefore if the producer may run on different architectures steps should be taken
to ensure that any endianness issues are handled correctly.

\subsubsection{Tags}\label{sec:Tags}
The \texttt{RawDataEvent} (in fact any type that descends from the \texttt{Event} base class)
may also have tags set.
These are name-value pairs containing extra information that does not easily fit
in the usual raw data.
This is used particularly in the \gls{BORE} to include information about the particular run
that may be useful for the decoding later.
A tag may be set as follows:
\begin{listing}
event.SetTag("Temperature", 42);
\end{listing}

The value corresponding to the tag can be set as an arbitrary type (in this case an integer),
it will be converted to a string internally.

\subsection{Log Messages}
A method is provided for sending log messages to the central Log Collector.
To use it the follwing header should be included:
\begin{listing}
#include "eudaq/Logger.hh"
\end{listing}

This defines the following macros for sending log messages,
listed in decreasing order of severity:
\begin{description}

\ttitem{EUDAQ\_USER}
A user-generated message (e.g. from the RunControl Log button).

\ttitem{EUDAQ\_ERROR}
Something that has gone wrong and should probably be looked into.

\ttitem{EUDAQ\_WARN}
A warning that something may not be quite right.

\ttitem{EUDAQ\_INFO}
An message generated during normal running containing information that may be useful to the user.

\ttitem{EUDAQ\_EXTRA}
Some extra information that may be less useful in normal running.

\ttitem{EUDAQ\_DEBUG}
Information for debugging purposes that will normally be hidden.

\end{description}

They are used as follows:
\begin{listing}
EUDAQ_ERROR("No keyboard detected: press F1 to continue.");
\end{listing}

The messages will be sent to the central Log Collector if it is connected,
otherwise they will be displayed on the local terminal.
The log level can be changed in the following way:
\begin{listing}
EUDAQ_LOG_LEVEL("WARN");
\end{listing}

Any messages lower than the specified level will just be ignored.
This can be useful to filter out unimportant messages and, for example, just display error messages.


\subsection{Interfacing Python-Code via the PyProducer Interface}
As described in section~\ref{sssec:pywrapper}, a Python interface is
provided for selected EUDAQ components including a Producer. This
basic implementation can be extended on the Python-side as
demonstrated by the example \texttt{python/example-producer.py}.
