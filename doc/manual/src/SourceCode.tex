% !TEX root = EUDAQUserManual.tex
\section{Source Code}
This section contains example code to illustrate the concepts in the manual,
when they are too long to be included in the main section.
%It also contains some header files from EUDAQ,
%in cases where they contain public interfaces that may be useful to the user.

All files are also present in the \texttt{EUDAQ} distribution;
so if possible those versions should be used, since they may be more up to date than the manual.

\subsection{Example Config File}\label{sec:ExampleConfig}
\myinputlisting[conf]{-30}{conf/ExampleConfig.conf}
\newpage

\subsection{Example Producer}\label{sec:ExampleProducer}
\myinputlisting{-60}{producers/example/src/ExampleProducer.cxx}
\newpage

\subsection{Example DataConverterPlugin}\label{sec:ExampleConverter}
\myinputlisting{-90}{producers/example/src/ExampleConverterPlugin.cc}
\newpage

%\subsection{ExampleHardware Header}\label{sec:ExampleHardwareHeader}
%This is not actually a real example to be followed,
%it is just so that the \texttt{ExampleProducer} compiles properly,
%and so that we can have some actual data to demonstrate the \texttt{ExampleConverterPlugin}.
%The header is included here as a reference since the interface is used in the ExampleProducer.
%
%\myinputlisting{-105}{main/include/eudaq/ExampleHardware.hh}
%\newpage

%\subsection{ExampleHardware Source}\label{sec:ExampleHardwareSource}
%As noted in \autoref{sec:ExampleHardwareHeader}, this is not a real example to be followed.
%
%\myinputlisting{-55}{main/src/ExampleHardware.cc}
%\newpage

\subsection{Example Reader}\label{sec:ExampleReader}
\myinputlisting{-50}{main/exe/src/ExampleReader.cxx}
\newpage

%\subsection{StandardEvent header}\label{sec:StandardEvent}
%\myinputlisting{-95}{main/include/eudaq/StandardEvent.hh}
