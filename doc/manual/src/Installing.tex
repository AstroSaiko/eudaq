% !TEX root = EUDAQUserManual.tex
\section{Installing EUDAQ}

The installation is described in four steps:%
\footnote{Quick installation instructions are also described on \url{http://eudaq.github.io/} or in the main \texttt{README.md} file of each branch, e.g. \url{https://github.com/eudaq/eudaq/blob/v1.6-dev/README.md}.}
\begin{enumerate}
\item Installation of (required) prerequisites
\item Downloading the source code (GitHub)
\item Configuration of the code (CMake)
\item Compilation of the code
\end{enumerate}

If you occur problems during the installation process, please have a look into the issue tracker on GitHub.%
\footnote{Go to \url{https://github.com/eudaq/eudaq/issues}} 
Here you can search, if your problem had already been experienced by someone else, or you can open a new issue (see \autoref{sec:reporting}).

\subsection{Installation of prerequisites}

EUDAQ has few dependencies on other software, but some features do rely on other packages:
\begin{itemize}
\item To get the code and stay updated with the central repository on GitHub git is used.
\item To configure the EUDAQ build process, the CMake cross-platform, open-source build system is used.
\item To compile EUDAQ from source code requires a compiler that implements the C++11 standard.
\item The libusb library is needed to communicate over USB with a \gls{TLU} \cite{Cussans2009}.
\item Qt is required to build GUIs of the e.g. Run Control Log Collector. 
\item ROOT is required for the Online Monitor.
\end{itemize}

\subsubsection{Git}
Git is a free and open source distributed version control and is available for all of the usual platforms \cite{gitWWW}. 
It allows for local version control and repositories, but also communicating with central online repositories like GitHub.
   
In order to get the EUDAQ code and stay updated with the central repository on GitHub git is used (see \autoref{sec:downloadingEUDAQ}).
But also for developing the EUDAQ code having different versions (tags) or branches (development repositories), git is used (see \autoref{sec:contributing}).


\subsubsection{CMake (required)}
In order to generate configuration files for building EUDAQ (makefiles) independently from the compiler and the operating platform, the CMake build system is used.

CMake is available for all major operating systems from \url{http://www.cmake.org/cmake/resources/software.html}. 
On most Linux distributions, it can usually be installed via the built-in package manager (aptitude/apt-get/yum etc.) and on OSX using packages provided by e.g. the MacPorts or Fink projects.

\subsubsection{C++11 compliant compiler (required)}
The compilation of the EUDAQ source code requires a C++11 complianti compiler and has been tested with GCC (at least version 4.8), Clang (at least version 3.1), and MSVC (Visual Studio 2012 and later) on Linux, OS X and Windows.

If you are using Scientific Linux, please install the \emph{Developer Toolset} available e.g. from \url{http://linux.web.cern.ch/linux/devtoolset/} to get access to a GCC version which fully implements C++11, e.g. on SL6 do
\begin{listing}[mybash]
scl use devtoolset-1.1 bash
\end{listing}
and cmake and install in this bash.

\subsubsection{libusb (for the EUDET TLU)}
In order to communicate over USB with a \gls{TLU}, the libusb library is needed.
Therefore, if you want to compile the \texttt{tlu} subdirectory, you should make sure that libusb is properly installed.

On Mac OS X, this can be installed using Fink or MacPorts.
If using MacPorts you may also need to install the \texttt{libusb-compat} package.

On Linux it may already be installed,
otherwise you should use the built-in package manager to install it.
Make sure to get the development version, which may be named \texttt{libusb-devel} instead of simply \texttt{libusb}, e.g. on Ubuntu: 
\begin{listing}[mybash]
sudo apt-get install libusb-dev 
\end{listing}

On Windows, libusb is only needed if compiling with cygwin,
in which case you should use the cygwin installer to install libusb.
Otherwise libusb is not needed, as the included ZestSC1 libraries should work as they are.

\subsubsection{ZestSC1 drivers and TLU firmware files (for the EUDET TLU)}
Additonally to the libusb library, the EUDET \gls{TLU} producer requires the
ZestSC1 driver package and the FPGA firmware bitfiles. 
These are available to download via AFS from DESY. 
If AFS is accessible on the machine when CMake is running, the necessary files will be installed automatically.
Otherwise, manually copy full folder with sub-directories from
\begin{itemize}
\item \texttt{/afs/desy.de/group/telescopes/tlu/ZestSC1} and
\item \texttt{/afs/desy.de/group/telescopes/tlu/tlufirmware}
\end{itemize}
into the \texttt{extern} subfolder in your EUDAQ source directory.


\subsubsection{Qt (for GUIs)}
The graphical interface of EUDAQ uses the Qt graphical framework.
In order to compile the \texttt{gui} subdirectory, you must therefore have Qt installed.
It is available in most Linux distributions as the package \texttt{qt4-devel} or  \texttt{qt5-devel}
but make sure the version is at least 4.4, since there are a few issues with earlier versions.

If the included version is too old, or on other platforms,
it can be downloaded from \url{http://qt.nokia.com/downloads}.
Select the LGPL (free) version, then choose the complete development environment
(it may also work with just the framework, but this is untested).
Make sure the \texttt{QTDIR} environment variable is set to the Qt installation directory,
and the \texttt{\$QTDIR/bin} directory is in your path.

If you are using OSX, the easiest way to install Qt is using the
packages provided by the MacPorts project (\url{http://www.macports.org/}).

\subsubsection{ROOT (for Monitor)}
\label{sec:Root}
The Online Monitor, as well as a few command-line utilities (contained in the \texttt{root} subdirectory), use the ROOT package for histogramming.
It can be downloaded from \url{http://root.cern.ch} or installed via
your favorite package manager.

Make sure ROOt's \texttt{bin} subdirectory is in your path, so that the \texttt{root-config} utility can be run.
This can be done by sourcing the \texttt{thisroot.sh} (or \texttt{thisroot.ch} for csh-like shells)
script in the \texttt{bin} directory of the ROOT installation:
\begin{listing}[mybash]
source /path-to/root/bin/thisroot.sh
\end{listing}

\subsubsection{LCIO / EUTelescope (for converting/analysis)}
\label{sec:LCIO-EUTel}
To enable the writing of \gls{LCIO} files, or the conversion of native files to \gls{LCIO} format,
EUDAQ must be linked against the \gls{LCIO} and EUTelescope libraries.
Detailed instructions on how to install both using the
\texttt{ilcinstall} scripts can be found at \url{http://eutelescope.web.cern.ch/content/installation}.

The \texttt{EUTELESCOPE} and \texttt{LCIO} environment variables should be set to the
installation directories of EUTelescope and LCIO respectively.
This can be done by sourcing the \texttt{build\_env.sh} script as follows:
\begin{listing}[mybash]
source /path-to/Eutelescope/build_env.sh
\end{listing}

%%%%%%%%%%%%%%%%%%%%%%%%%%%%%%%%%%%%%%%%%%%%%%%%%%%%%%%%%%%
\subsection{Download the source code from GitHub}
\label{sec:downloadingEUDAQ}

The EUDAQ source code is hosted on GitHub \cite{githubEUDAQ}. 
Here, we describe how to get the code and install a stable version release. 
In order to get information about the work flow of developing the EUDAQ code, please find the relevant information in see \autoref{sec:contributing}.

\subsubsection{Downloading the code (clone)}
We recommend to obtain the software by using git,
since this will allow you to easily update to newer versions.
The source code can be downloaded with the following command:
\begin{listing}[mybash]
git clone https://github.com/eudaq/eudaq.git eudaq
\end{listing}
This will create the directory \texttt{eudaq}, and download the latest
version into it. 

\textit{Note:} Alternatively and without version control, you can also download a zip/tar.gz file of EUDAQ releases (tags) from \url{https://github.com/eudaq/eudaq/releases}. 
By downloading the code, you can skip the next two subsections. 

\subsubsection{Changing to a release version (checkout)}
After cloning the code from GitHub, your local EUDAQ version is on the master branch (check with \texttt{git status}).  
For using EUDAQ without development or for production environments (e.g. at test beams), we strongly recommend to use the latest release version. 
Use 
\begin{listing}[mybash]
git tag 
\end{listing}
in the repository to find the newest stable version as the last entry.
In order to change to this version in your local repository, execute e.g. 
\begin{listing}[mybash]
git checkout v1.6.0
\end{listing}
to change to version v1.6.0.

\subsubsection{Updating the code (fetch)}
If you want to update your local code, e.g to get the newest release versions, execute in the \texttt{eudaq} directory: 
\begin{listing}[mybash]
git fetch
\end{listing}
and check for new versions with \texttt{git tag}. 


\subsection{Configuration via CMake}
\label{sec:cmake}
CMake supports out-of-source configurations and generates building files for compilation (makefiles). 
Enter the \texttt{build} directory and run CMake, i.e.
\begin{listing}[mybash]
cd build
cmake ..
\end{listing}
CMake automatically searches for required packages and verifies that all dependencies are met using the \texttt{CMakeLists.txt} scripts in the main folder and in all sub directories. 

By default, only the central shared library, the main executables and (if Qt4 or Qt5 have been found) the graphical user interface (GUI) are configured for compilation. 
You can modify this default behavior by passing the \texttt{-DBUILD\_[name]} option to
CMake where \texttt{[name]} refers to an optional component, e.g.
\begin{listing}[mybash]
cmake -DBUILD_gui=OFF -DBUILD_tlu=ON ..
\end{listing}
to disable the GUI but enable additionally executables of the TLU producer.
Find some of the most important building options in \autoref{tab:cmakeoptions}.

\begin{table}[!h]
{\footnotesize
\begin{tabular}{l|l|p{9.5cm}}
option &  default &  comment \\
\hline
\texttt{-DBUILD\_main} &  ON & Builds main EUDAQ executables.
The common library, and some command-line programs that depend on only this library \\
\texttt{-DBUILD\_manual} & OFF &  Builds Manual in pdf-format. In Ubuntu e.g.: \texttt{pdflatex}, \texttt{scrartcl.cls} and \texttt{upquote.sty} are required, thus execute \texttt{sudo apt-get install texlive-latex-base texlive-latex-recommended texlive-latex-extra} \\
\texttt{-DBUILD\_tlu} &  OFF &  Builds the TLU producer and command-line tools. libus, ZestSC1 and tlufirmware files are required. \\
\texttt{-DBUILD\_<producername>} &  OFF &  \texttt{-DBUILD\_<producername>=ON} is needed to enable bulding of a specific producer. 
The producername is the same as the name of the producer's directory in \texttt{./producer}. 
These are user-contributed producers for specific detectors operating with the EUDET-type beam telescope. They should not be compiled unless needed.\\
\texttt{-DBUILD\_gui} &  ON & Builds GUI executables, such as the Run Control and Log Collector. Requires QT4/5. \\
\texttt{-DBUILD\_python} & ON & Builds Python EUDAQ binding library. \\
\texttt{-DBUILD\_pybindgen} &  OFF &  Builds pybindgen binding libraries. \\
\texttt{-DBUILD\_onlinemon} &  ON & Builds Online Monitor executable. Requires ROOT. \\
\texttt{-DBUILD\_offlinemon} & OFF &  Builds offline monitor executable. Requires ROOT. \\
\texttt{-DBUILD\_metamon} &  OFF &  Builds MetaData Monitor executable. \\
\texttt{-DBUILD\_resender} & OFF &  Builds resender producer. \\
\texttt{-DBUILD\_nreader} &  OFF &  Builds native reader Marlin processor used for data conversion into LCIO. Requires LCIO/EUTelescope \\
\texttt{-DINSTALL\_PREFIX=<PATH>} & \texttt{<eudaq>} & In order to install the executables into \texttt{bin} and the library into \texttt{lib} of a specific \texttt{<path>}, instead of into the \texttt{<eudaq>} path. 
The corresponding de-installation can be done by:
\texttt{cd <eudaq>\/build\/ \&\& sudo xargs rm < install\_manifest.txt}
\end{tabular}
\caption{Some of the most important building options for CMake.}
\label{tab:cmakeoptions}
}
\end{table}

\textit{Note:} After generating building files by running \texttt{cmake ..}, you can list all possible option and their status by running \texttt{cmake -L}. 
Using a GUI version of CMake shows also all of the possible options.   

Corresponding settings are cached, thus they will be used again next time CMake is running.
If you encounter a problem during installation, it is recommended to clean the cache by just removing all files from the build folder, since it only contains automatically generated files. 
In order to start from scratch, just run:
\begin{listing}[mybash]
cd build
rm -rf *
\end{listing}


%%%%%%%%%%%%%%%%%%%%%%%%%%%%%%%%%%%%%%%%%%%%%%%%%%%%%%%%%%%%%%%55
\subsection{Compilation}

\subsubsection{Compilation on Linux/OSX}
From the top EUDAQ directory, run the command
\begin{listing}[mybash]
cd build
make install
\end{listing}
in order to compile the common library with some command-line programs (the contents of the \texttt{./main/exe} subdirectory).
If other parts are needed, you can specify them as arguments to the
CMake command during the configuration step (see \ref{sec:cmake}).

The executable binaries and the common shared library will be installed by default into the
\texttt{bin} and \texttt{lib} directories in the source tree,
respectively. If you would like to install into a different location,
please set the respective parameter during the CMake configuration.

\subsubsection{Setup and Compilation on Windows using Visual~Studio}

This section gives a short overview on the steps needed to compile the
project under Windows (tested under Windows 7, 32-bit and 64-bit). For a more
detailed introduction to the Windows build system and Visual~Studio
project files see the appendix~\ref{app:compileOnWindows} on
page~\pageref{app:compileOnWindows}.

\begin{itemize}
\item Prerequisites 
\begin{itemize}
\item Compiler: Download Visual Studio Express Desktop 2012 or later (e.g. 2013 Version):
  \url{http://www.microsoft.com/en-us/download/details.aspx?id=40787}
\item Download Qt4 or later Qt5
\item Download and install the pthreads library (pre-build binary from
  \url{ftp://sources.redhat.com/pub/pthreads-win32}) into either
  \texttt{\lstinline{c:\\pthreads-w32}} or \texttt{./extern/pthreads-w32}
\end{itemize}

\item Start the Visual Studio \emph{Developer Command Prompt} from the
  Start~Menu entries for Visual~Studio (Tools subfolder) which opens a
  \texttt{cmd.exe} session with the necessary environment variables
  already set. If your Qt installation has not been added to the
  global \texttt{\%PATH\%} variable, you need to execute the \texttt{qtenv2.bat} batch file (or similar) in the Qt folder, e.g.
  \begin{listing}[mybash]
C:\Qt\Qt5.1.1\5.1.1\msvc2012\bin\qtenv2.bat
\end{listing}
Replace "5.1.1" with the version string of your Qt installation.

\item Now clone the EUDAQ repository (or download using GitHub) and enter the build directory on the prompt, e.g. by entering
  \begin{listing}[mybash]
cd c:\Users\[username]\Documents\GitHub\eudaq\build
\end{listing}

\item Configuration: Now enter
  \begin{listing}[mybash]
cmake ..
\end{listing}
to generate the VS project files.

\item Compile by calling
  \begin{listing}[mybash]
MSBUILD.exe EUDAQ.sln /p:Configuration=Release
\end{listing}
or install into \texttt{\lstinline{eudaq\\bin}} by running
  \begin{listing}[mybash]
MSBUILD.exe INSTALL.vcxproj /p:Configuration=Release
\end{listing}
\item This will compile the main library and the GUI. For additional processors, please check the individual documentation.
\end{itemize}

Note on ``\emph{moc.exe - System Error: The program can't start
  because MSVCP110.dll is missing from your computer}'' errors: when using Visual~Express~2013 and \texttt{pthreads-w32} 2.9.1, you might require ``Visual C++ Redistributable for Visual Studio 2012'': download (either x86 or x64) from \url{http://www.microsoft.com/en-us/download/details.aspx?id=30679} and install.

