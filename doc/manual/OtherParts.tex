% !TEX root = EUDAQUserManual.tex
\section{Other Parts of the Framework}
The EUDAQ framework contains a number of other parts that may be useful.
Those that have not already been described in previous sections will be outlined below.

\subsection{FileWriter}
The FileWriter part of the framework allows different file formats to be written,
using a common interface, using a plugin-like system to define new file types.
The \texttt{FileWriter} class defines the interface that each type must implement,
and the \texttt{FileWriterFactory} allows code that writes data files to select any
available file type, and write it in a generic way,
without needing to know details about the particular file format.
A number of different file types are already implemented,
for a list with descriptions, see page \pageref{lst:FileTypes}

The easiest way to make use of the different \texttt{FileWriter}s,
is to use the \texttt{Converter.exe} program (see \autoref{sec:Converter}).

The \texttt{FileWriter} base class is defined in the following header:
\begin{listing}
#include "eudaq/FileWriter.hh"
\end{listing}

In order to implement a new \texttt{FileWriter}, a new class must be written,
inheriting from the \texttt{FileWriter} base class, and implementing the following methods:
\begin{listing}
virtual void StartRun(unsigned);
virtual void WriteEvent(const DetectorEvent &);
virtual unsigned long long FileBytes() const;
\end{listing}

The \texttt{StartRun} method is called at the start of each new run
with the run number as a parameter.
This allows a new file to be opened, and any header information to be written if necessary.
Then the \texttt{WriteEvent} method is called for each event to be written.
Here the \texttt{DetectorEvent} can be decoded and processed
and the necessary data written to file.
The \texttt{FileBytes} method should return the number of bytes written to the file.
However, it is optional, and may simply return zero if the actual size is not easily known.

\subsection{FileReader}
Although tools are provided to access the information in the native data files,
and to convert them to other formats (such as LCIO, for analysis with the EUTelescope package),
in some cases it may be preferable to access the native data directly.
For this, the \texttt{FileReader} class is provided,
allowing a custom program to be written to access a native file and process it as desired.

The constructor takes as an argument the name of the file to be opened,
and will read the first event from the file (which should be the \gls{BORE}).
The \texttt{NextEvent()} method can then be called to advance through the file.
It can optionally take as a parameter the number of events to skip,
and will return \texttt{true} as long as a new event was read.
The currently loaded event can be accessed with the \texttt{GetDetectorEvent()} method.

The basic usage is shown below, while a more complete example is available in \autoref{sec:ExampleReader}:
\begin{listing}
#include "eudaq/FileReader.hh"
#include <iostream>

int main(int argc, char ** argv) {
  if (argc != 2) {
    std::cerr << "usage: " << argv[0] << " file" << std::endl;
    return 1;
  }
  eudaq::FileReader reader(argv[1]);
  std::cout << "Opened file: " << reader.Filename() << std::endl;
  std::cout << "BORE:\n" << reader.GetDetectorEvent() << std::endl;
  while (reader.NextEvent()) {
    std::cout << reader.GetDetectorEvent() << std::endl;
  }
  return 0;
}
\end{listing}

This will open the file specified on the command line,
and print out a summary of all the events in there.
Be aware that running it as it is may generate a large amount of output,
especially with large data files.

%\subsection{FileNamer}
\subsection{PluginManager}\label{sec:PluginManager}
The \texttt{PluginManager} handles the different \texttt{DataConverterPlugin}s,
allowing raw data stored in a \texttt{RawDataEvent} to be easily converted
to a \texttt{StandardEvent} or \texttt{LCEvent} without having to know the details of all the detector types in there.
It is defined in the following header:
\begin{listing}
#include "eudaq/PluginManager.hh"
\end{listing}

In order to convert the events correctly,
the plugins must have access to the information in the BORE.
Therefore, before any events may be converted, and for each data file,
the \texttt{PluginManager} must be initialized as follows:
\begin{listing}
eudaq::PluginManager::Initialize(bore);
\end{listing}

The \texttt{PluginManager} will take care of passing the relevant parts of the \gls{BORE}
to the appropriate \texttt{DataConverterPlugin}s.
The \texttt{DetectorEvent}s can then be converted as follows:
\begin{listing}
eudaq::StandardEvent sev = eudaq::PluginManager::ConvertToStandard(dev);
\end{listing}

The \texttt{PluginManager} will take care of splitting the \texttt{DetectorEvent}
into its constituent subevents, and passing them all to the appropriate
\texttt{DataConverterPlugin}s to be inserted into the returned \texttt{StandardEvent}.
For a slightly more complete example of the \texttt{PluginManager},
see the \texttt{ExampleReader} in \autoref{sec:ExampleReader}.

\subsection{OptionParser}
The \texttt{OptionParser} is used to simplify parsing of command-line options.
It provides a way to specify which arguments a program accepts,
with the types, default values and descriptions,
so that the help text can be automatically generated,
and therefore is always in sync with the code,
and all command line programs can have a uniform interface.

All programs using the \texttt{OptionParser} will automatically provide a
\texttt{-h} (and \texttt{--help}) option to display the help text,
as well as a \texttt{-v} (and \texttt{--version}) option to display the program version,
unless the program explicitly overrides these options with other ones with the same names.

The \texttt{OptionParser} is the class that handles the actual parsing of the command line.
The signature of the constructor is as follows:
\begin{listing}
OptionParser(const std::string & name, const std::string & version,
             const std::string & desc="", int minargs = -1, int maxargs = -1);
\end{listing}
The first three arguments are the program name, version and (optionally) description,
and these are optionally followed by two numbers specifying the number of arguments
expected after the command line options.
The default value of -1 for the minimum means no arguments are allowed,
and for the maximum means that an arbitrary number may be given
(i.e. there is no explicit maximum).

If the automatically generated help text is not sufficient,
extra text may also be given to display at the end of the help text,
by passing it to the following method:
\begin{listing}
void OptionParser::ExtraHelpText(const std::string & text);
\end{listing}
This can be used to provide extra information about the options to the program.

Once an \texttt{OptionParser} object has been constructed, the different options may be specified.
There are two types: \texttt{OptionFlag}, which specifies a simple option with no argument,
and the template \texttt{Option<T>}, which specifies an option taking an argument of type \texttt{T}.

The \texttt{OptionFlag} constructor has the following signature:
\begin{listing}
OptionFlag(OptionParser & op, const std::string & shortname,
           const std::string & longname, const std::string & desc = "");
\end{listing}
where \texttt{op} is a reference to the \texttt{OptionParser} object created previously,
that will do the actual parsing of the command line.
It then takes two names: a short version (usually a single character) that is used with a single hyphen,
and a long version that must be preceded by two hyphens on the command line.
Finally, a description may be given that will be displayed in the help text.

The \texttt{Option} constructor has the following two signatures,
one for normal types, the other for vectors of another type:
\begin{listing}
Option<T>(OptionParser & op, const std::string & shortname,
          const std::string & longname, const T & deflt = T(),
          const std::string & argname = "", const std::string & desc = "");
Option<std::vector<T> >(OptionParser & op, const std::string & shortname,
          const std::string & longname, const std::string & argname = "",
          const std::string & sep = "", const std::string & desc = "");
\end{listing}
where, in both cases, the first three arguments are as for \texttt{OptionFlag}.
The first constructor then takes a default value that will be used
in the case the option is not specified on the command line,
a name for the argument to the option (to be used in the help text),
and a description of the option.
The vector version also takes an argument name and a description,
but no default value (the default is always an empty vector), instead it takes a separator,
which is the string used to separate multiple elements of the vector on the command line.
By default (or if an empty string is specified), a comma will be used.

Once all the options have been specified, the command line can be parsed,
which is done by calling the following method of the \texttt{OptionParser} object:
\begin{listing}
OptionParser & OptionParser::Parse(const char ** args);
\end{listing}
as an argument it takes the list of arguments from the command line (by convention usually called \texttt{argv}).
If there is an error during parsing, an exception may be thrown;
this should be handled by the \texttt{HandleMainException} method as described below.

Afterwards the values of the options can be accessed using their \texttt{Value()} method.
The \texttt{IsSet()} method is also available to tell whether an option has been set on the command line
(for \texttt{OptionFlag}s this will hold the same value as the \texttt{Value()} method).

Finally, the \texttt{OptionParser} has a \texttt{HandleMainException} method that provides
a way to catch any unhandled exceptions,
and either display help if it is a problem with parsing the command line,
or otherwise display a standard text informing the user of a problem.
It will also catch exceptions of type \texttt{MessageException} and display the message,
without treating it as an error, so this can be used to exit the program with a message to the user.
It is recommended to put the main program inside a \texttt{try} block,
then call the \texttt{HandleMainException} method from a \texttt{catch(...)} block,
after any other exceptions have been handled (if necessary).

An example use is shown below, illustrating most of what is described above:
\lstinputlisting[style=full, language=C++]{../../main/src/OptionExample.cxx}

Running this program produces the following output:
\begin{listing}[mybash]
$[./OptionExample.exe]$ -h
Example version 1.0
An example program

usage: ./OptionExample.exe [options] [0 or more arguments]

options:
  -t --test
     Enable test
  -e --example <value>	(default = 42)
     Example parameter
  -a --another <values>	(default = )
     Example vector

Some more information about this program.

$[./OptionExample.exe]$
Test: Disabled
Example: 42
Another: 
No arguments were given

$[./OptionExample.exe]$ -t -e 2.718 -a 1;2;3 foo bar
Test: Enabled
Example: 2.718
Another: 1, 2, 3
Argument 1: foo
Argument 2: bar
\end{listing}

\subsection{Timer}
The \texttt{Timer} class wraps the underlying operating system's timer functions,
making them easier to use in a platform independent way.
Whenever a \texttt{Timer} object is created, it will record the current time.
Then at any time in the future, the elapsed time in seconds may be accesses
with the \texttt{Seconds()} method.

There is also a \texttt{Stop()} method to stop the timer counting, so any subsequent calls
to \texttt{Seconds} will return the same value, and a \texttt{Restart()} method to
reset the timer's start time to the current time and start counting again.
An example use is shown below:
\begin{listing}[C++]
#include "eudaq/Timer.hh"

Timer t;
function_a();
cout << "Function A took " << t.Seconds() << " seconds." << endl;
t.Restart();
function_b();
cout << "Function B took " << t.Seconds() << " seconds." << endl;
// wait 3 microseconds
t.Restart();
while (t.Seconds() < 3e-6) {
  // do nothing
}
\end{listing}

This shows a timer being used to measure the execution time of two functions,
and to wait for a small delay.
Usually to wait for a delay, it is preferable to use sleep (or \texttt{mSleep}, see \autoref{sec:mSleep}),
but in most operating systems the minimum delay for a sleep is around 20~ms
(even when using \texttt{usleep} which has microsecond resolution)
so if the delay must be shorter, a busy loop like above is needed.

\subsection{Utils}
The \texttt{Utils} package is a collection of useful functions and classes too small to merit
their own individual files.
It is used by including the header:
\begin{listing}
#include "eudaq/Utils.hh"
\end{listing}

Some of the most useful parts are described here.

\subsubsection{to\_string}
This is a template function that takes (almost) any type and returns the value converted to a string.
An optional second argument specifies the minimum number of digits to use
(padding with zeroes if necessary).
\begin{listing}
int value = 123;
strfunction(to_string(value));
strfunction(to_string(value, 6));
\end{listing}

This will pass first the string \inline{"123"},
and then the string \inline{"000123"}
to the function \texttt{strfunction}.

\subsubsection{from\_string}
This template function is the inverse of \texttt{to\_string}.
It takes as arguments a string and a default value of type T,
and returns an object of type T initialised from the string.
If it is not possible to convert the string to the required type,
the default value is returned instead.
\begin{listing}
std::string value = "456";
intfunction(from_string(value, 0));
\end{listing}

This will call \texttt{intfunction} with the integer value 456.

\subsubsection{hexdec}
This is a class to facilitate printing numbers in both hexadecimal and decimal.
It is used similarly to \texttt{to\_string}, but when printed,
it will display the value in hexadecimal, followed by the value in decimal in parentheses.
The hexadecimal values will be padded to the full width of the type,
unless a second argument is given specifying the minimum number of hex digits to display.
\begin{listing}
short value = 789;
cout << hexdec(value) << endl
     << hexdec(value, 0) << endl;
\end{listing}

This will display:
\begin{listing}[]
0x0315 (789)
0x315 (789)
\end{listing}

If the result is required in a string, instead of being printed,
this can be achieved with \inline{to_string(hexdec(value))}.

\subsubsection{mSleep}\label{sec:mSleep}
This is a wrapper around the operating system's \texttt{sleep}/\texttt{usleep}
(or equivalent) function.
It takes as an argument the number of milliseconds to sleep.
The advantage of this function is that it will work on Linux,
Mac OS X and Windows, as it will automatically call the correct underlying function.

%\section{Monitoring and Analysis}
%The EUDAQ framework is does not perform any detailed analysis of the data taken.
%For this, the EUTelescope\cite{eutel2008} package is available.
%In order to use it, the data must be converted to LCIO format (\autoref{sec:LCIO}).
